\documentclass[11pt]{article}
\usepackage[utf8]{inputenc}
\usepackage{fullpage}
\usepackage[T1]{fontenc}
\usepackage[francais]{babel}
\usepackage{listings}\lstset{basicstyle=\sffamily,columns=fullflexible,language=C}
\title{Jeu des sept couleurs: Implémentation et réalisation d'une IA}
\author{Antonin Garret & Jimmy Rogala}
\date{Fevrier-Mars 2016}
\begin{document}
  \maketitle
  \section*{Introduction}
    Les règles du jeu des 7 couleurs sont:
    \begin{itemize} %il faut rajouter des images
      \item Un tableau d'une certaine taille est rempli de 7 couleurs aléatoirement
      \item La case en bas à gauche (resp: en haut à droite) est de la couleur 'v' (resp: '^') du joueur 0 (resp: 1)
      \item Chaque tour le joueur 0 (resp: 1 ) choisi une couleur parmi les 7 couleurs. Toute les cases de la couleur choisie qui sont juxtaposées à une case du joueur 0 (resp: 1) directement ou indirectement prennent la couleur du joueur 0 (resp: 1).
      \item Le jeu termine quand un joueur a rempli la majorité du tableau: il est le gagnant.
    \end{itemize}
    Nous allons d'abord aborder la réalisation du jeu en lui-même (et des choix de l'implémentation) puis parler de la réalisation de plusieurs IA pour ce même jeu.
  \section{Réalisation du jeu}
    Le langage imposé fut le C. Il permet un contrôle rigoureux de la mémoire.
    \subsection{Implémentation}
      Le choix le plus intuitif était d'utiliser un tableau pour représenter le plateu de jeu. C'est ce que nous avons fait. L'autre choix cohérent et possible consistait à définir les cases en tant qu'objet et les voisins des cases mais le C n'est pas pratique pour faire de la programmation orientée objet.
      Nous avons fait le choix de représenter en interne les couleurs par les nombre de 2 à 9, et les joueurs 0 et 1 par le nombres 0 et le nombre 1. Des fonctions de traduction permettent ensuite d'afficher les couleurs correspondantes pour l'utilisateur.

      Les fonctions importantes pour la réalisation d'un tour de jeu sont:
      \subsubsection{initgame}
        initgame() consiste à initialiser le plateau de jeu. On rempli le tableau de chiffres de 2 à 9 aléatoirement, puis la case en bas à gauche (resp: en haut à droite) de la couleur du joueur 0 (resp: joueur 1). La fonction est en $0(n)$ (on notera à partir de maintenant $n$ le nombre de case du tableau)
\end{document}
