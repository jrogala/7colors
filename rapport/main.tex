\documentclass{article}
\usepackage[utf8]{inputenc}
\usepackage[T1]{fontenc}
\usepackage[francais]{babel}
%\usepackage{hyperref}

\title{Jeu des sept couleurs: Implémentation et réalisation d'une IA}
\author{Antonin Garret & Jimmy Rogala}
\date{Mars 2016}
\begin{document}
\maketitle
\tableofcontents
\clearpage
\section{Le jeu}
Nous allons implementer le jeu des 7 merveilles du monde en C.

  \subsection{Les règles du jeux}
    Les règles du jeu des 7 couleurs sont:
    \begin{itemize} %il faut rajouter des images
      \item Un tableau d'une certaine taille est rempli de 7 couleurs aléatoirement.
      \item La case en bas à gauche (resp: en haut à droite) est de la couleur v (resp: \textasciicircum ) du joueur 0 (resp: 1).
      \item Chaque tour le joueur 0 (resp: 1 ) choisit une couleur parmi les 7 couleurs. Toute les cases de la couleur choisie qui sont juxtaposées à une case du joueur 0 (resp: 1) directement ou indirectement prennent la couleur du joueur 0 (resp: 1).
      \item Le jeu termine quand un joueur a rempli la majorité du tableau: il est le gagnant.
    \end{itemize}

    Nous allons d'abord aborder la réalisation du jeu en lui-même (et des choix de l'implémentation) puis parler de la réalisation de plusieurs IA pour ce même jeu.


\section{Voir le monde en 7 couleurs}
    Le langage imposé fut le C. Il permet un contrôle rigoureux de la mémoire.
    \subsection{Question 2.1: Remplir le monde}
      Le choix le plus intuitif était d'utiliser un tableau pour représenter le plateu de jeu. C'est ce que nous avons fait. L'autre choix cohérent et possible consistait à définir les cases en tant qu'objet et les voisins des cases mais le C n'est pas pratique pour faire de la programmation orientée objet.
      Nous avons fait le choix de représenter en interne les couleurs par les nombre de 2 à 9, et les joueurs 0 et 1 par le nombres 0 et le nombre 1. Des fonctions de traduction permettent ensuite d'afficher les couleurs correspondantes pour l'utilisateur.

La fonction \textit{initgame()} permet d'initialiser le plateau de jeu. On remplit le tableau de chiffres de 2 à 9 aléatoirement, puis la case en bas à gauche (resp: en haut à droite) de la couleur du joueur 0 (resp: joueur 1). La fonction est en $\mathcal{O}(n)$ (on notera à partir de maintenant $n$ le nombre de case du tableau)

    \subsection{Question 2.2: Jouer un coup}
    Nous avons réaliser la méthode proposée pour jouer un coup. Elle consiste à parcourir le tableau de jeu et à trouver toute les cases de la couleur choisie par le joueur actuel qui sont adjacentes à l'une de ses cases. Si on a mis à jour au moins une case au cours de ce parcours, on effectue un nouveau parcours. On s'arrête lorqu'on à effectué un parcours complet sans changement. Le moyen le plus simple pour tester la correction de cette méthode est de l'appliquer quelques fois sur le tableau de jeu en affichant celui-ci et de vérifier que tout se passe comme prévu.

    Dans le pire des cas, on effectuera $n-2$ parcours (si l'on modifie exactement une case à chaque parcours). Comme chaque parcours vérife les $n$ cases du tableau, on a une complexité dans le pire des cas de $\mathcal{O}(n^2)$.

    La plupart du temps on effetuera que quelques parcours (car il faut une grande zone de même couleur pour effectuer de nombreux parcours, ce qui est statistiquement peu probable), on a donc une complexité moyenne de $\mathcal{O}(n)$. Ce pendant, il est possible de faire mieux

    \subsection{Question 2.3: Jouer un coup : meilleure alternative}
    La methode précedente est peu efficace. La methode choisi consiste à faire comme un front d'onde: On part de la case du joueur puis on regarde ses voisins. Si les voisins sont de la couleur du joueur, on réapplique la fonction sur les voisins. Si les voisins sont de la couleur choisi par le joueur, on change la couleur de la case et on vérifie si la case a des voisins de la même couleur pour réappliquer l'algorithme sur les voisins. Nous avons utiliser un autre tableau pour retenir les cases déjà parcouru donc on a une compléxité spatialle en $\mathcal{O}(n)$ permettant une compléxité temporelle en $\mathcal{O}(CaseDuJoueurApresCoup)$.

    Un moyen de vérifier si la fonction est correcte est de print le board_temp (notre tableau permettant d'éviter les boucles infinis). Si celui ci est en adéquation avec l'état du jeu, alors la fonction est probablement correcte.

    \subsection{Question 2.3: Autre possibilité}
    On aurait pu (dans l'idée du pot de peinture dans paint) pendre les cases du joueur en la couleur voulu, puis repeindre toutes les cases de la couleur du joueur. Ceci evite la compléxité spatialle en $\mathcal{O}(n)$

    \section{A la conquête du monde}

    Le jeu a été fait en sorte que le changement de l'algorithme (IA simple, IA compliqué ou joueur) soit simple. Il nous suffit donc de créer une fonction qui renvoit une couleur en prenant en argument le joueur qu'il est (0 ou 1)

    \subsection{Question 3.1: Joueur vs Joueur}

    Une simple boucle qui vérifie que le joueur a bien rentré un nombre permet à un joueur de jouer. On a pas vidé le buffer entre chaque coup donc un joueur peut jouer pour l'autre joueur.

    \subsection{Question 3.2: Condition d'arrêt}

    La condition d'arrêt est levé si un joueur possede plus de 50 \% du terrain.

    \section{La stratégie de l'aléa}

    \subsection{Question 4.1 et 4.2: Un début d'IA}

    L'IA la plus simple reste l'aléatoire, la premiere fonction est simple à faire. La deuxieme est plus compliqué mais il suffit de garder en mémoire les couleurs quand on passe dessus.

    \section{La loi du plus fort}

    \subsection{Question 5.1: Enfin une bonne IA}

    Le passage de l'IA aléatoire pas trop bête à cette fonction est simple: Au lieu de seulement mémoriser les couleurs, on mémorise le nombre de fois qu'on voit la couleur. Une attention est porté au fait que on doit vérifier les voisins de chaque case coloré (comme pour jouer un coup). La compléxité est la même que pour jouer un coup.

    \subsection{Question 5.2: Un monde équitable?}

    Quand on fait jouer l'aléatoire contre le glouton, le glouton l'emporte souvent. Quand on fait le glouton contre le glouton, le premier joueur l'emporte le plus souvent (avec une différence moindre par rapport à Aléa VS Glouton). Les idées pour rendre le jeu équilibré est de faire un monde symetrique (symetrie axialle par rapport à la diagonale séparant les 2 joueurs) et d'alterner le joueur 1 et le joueur 2. Ceci est réalisable sur plusieurs parties mais pour ne privilieger aucun joueur sur une seule partie, l'idée a été de forcer un nombre de case proche pour chaque couleur (qu'il y ait autant de rouge que de vert que de bleu etc). Ceci empecherait le fait qu'un joueur domine car il commence (mais ceci ne garantie rien, l'algorithme étant aléatoire).

    \section{Les nombreuses huitièmes merveilles du monde}

    Les questions suivantes n'ont pas été implémentées. Une expérience de pensée est fait dessus.

    \subsection{Question 6.1 et 6.2: Un début de Min-Max}

    La meilleur façon de faire les questions suivantes et de penser en termes d'heuristique et d'arbre: L'hégémonique est comme le glouton, sauf que l'heuristique n'est pas la même. Il suffit de créer une fonction qui calcule le périmetre et de l'utiliser dans un min max de hauteur 1 pour avoir la réponse à la 6.1 (il suffit d'utiliser une hauteur de 2 pour avoir la réponse à la 6.2). La compléxité de l'algorithme pour une prévision sur les k prochains coup est en $\mathcal{O}(7^k)$ appel de la fonction coup.

    \section*{Synthèse du projet}

    \subsection*{Observation diverse}

    Les résultats sont bien en adéquation avec l'intuition (contrairement à certains autres jeux).

    \subsection*{Ce qui a marché!}

    Le début de code permet de commencer directement la partie interessante. Le jeu en lui même est suffisament simple pour pouvoir finir l'implémentation en 1-2h (pas comme le jeu de dames ou d'echec) et suffisament compliqué pour pouvoir dire des choses dessus (Le morpion est trop simple).

    \subsection*{Les couacs}

    NULL

    \subsection*{Ce qu'on peut ajouter}
    \begin{itemize}
    \item Pour donner envie de s'attaquer aux problèmes 7.x pourquoi ne pas mettre en place un concours en ligne d'ia sous la forme d'un tournoi mis à jour chaque 5 minutes entre tous nos algos? (La méthode a été utilisé en sup dans ma prépa et avait eu un franc succés)

    \item Le projet pourrait être tout simplement posé sous la forme de question: Mettre au point une démarche pour avoir un algorithme efficace dans le jeu des 7 couleurs (Nous expliquer seulement les règles du jeu). Ceci pourrait nous forcer à recréer le jeu pour tester nos algorithmes.


    \end{itemize}
\end{document}
