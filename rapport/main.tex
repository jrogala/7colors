\documentclass[11pt]{article}
\usepackage[utf8]{inputenc}
\usepackage{fullpage}
\usepackage[francais]{babel}
\usepackage{listings}\lstset{basicstyle=\sffamily,columns=fullflexible,language=C}
\title{Jeu des sept couleurs: Implémentation}
\author{Antonin Garret & Jimmy Rogala}
\date{Fevrier-Mars 2016}
\begin{document}
  \maketitle
  \section*{Introduction}
    Les règles du jeux des 7 couleurs sont:
    \begin{itemize} %il faut rajouter des images
      \item Un tableau d'une certaine taille est rempli de 7 couleurs aléatoirement
      \item La case en bas à gauche (resp: en haut à droite) est de la couleur du joueur 0 (resp:1)
      \item Chaque tour le joueur 0 (resp: 1 ) choisi une couleur parmi les 7 couleurs. Toute les cases de la couleur choisi qui sont juxtaposées à une case du joueur 0 (resp: 1) directement ou indirectement devient de la couleur du joueur 0 (resp: 1).
      \item le jeu termine quand un joueur à rempli la majorité du tableau: il est le gagnant.
    \end{itemize}
    Nous allons vous parler de la réalisation du jeu en lui même (et des choix de l'implémentation) puis vous parler de la réalisation d'IA pour ce même jeu.
  \section{Réalisation du jeu}
    Le langage imposé fut le C. Il permet un controle rigoureux de la mémoire.
    \subsection{Implémentation}
      Le choix d'utiliser un tableau trivial. L'autre choix cohérent et possible consistait à définir les cases en tant qu'objet et les voisins des cases mais le C n'est pas pratique pour faire de l'orientés objet.
      Nous avons fait le choix aussi d'utiliser les chiffres de 2-9 pour representer les couleurs et les chiffres 0 et 1 pour representer le joueur 0 et le joueur 1. Notre programme est aussi compatible pour un nombre quelqu'onc de couleur
      Les fonctions importante pour la réalisation d'un tour de jeu sont:
      \subsubsection{initgame}
        initgame() consiste à initialiser le plateau de jeu. On rempli le tableau de chiffres de 2-9 aléatoirement, puis la case en bas à gauche (resp: en haut à droite) de la couleur du joueur 0 (resp: joueur 1). La fonction est en $0(n)$ (n est (et sera) le nombre de case du tableau)
\end{document}
